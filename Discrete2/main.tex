%\iffalse
\let\negmedspace\undefined
\let\negthickspace\undefined
\documentclass[journal,12pt,twocolumn]{IEEEtran}
\usepackage{cite}
\usepackage{amsmath,amssymb,amsfonts,amsthm}
\usepackage{algorithmic}
\usepackage{graphicx}
\usepackage{textcomp}
\usepackage{xcolor}
\usepackage{txfonts}
\usepackage{listings}
\usepackage{enumitem}
\usepackage{mathtools}
\usepackage{gensymb}
\usepackage{comment}
\usepackage[breaklinks=true]{hyperref}
\usepackage{tkz-euclide} 
\usepackage{listings}
\usepackage{gvv}                                        
\def\inputGnumericTable{}                                 
\usepackage[latin1]{inputenc}                                
\usepackage{color}                                            
\usepackage{array}                                            
\usepackage{longtable}                                       
\usepackage{calc}                                             
\usepackage{multirow}                                         
\usepackage{hhline}                                           
\usepackage{ifthen}                                           
\usepackage{lscape}

\newtheorem{theorem}{Theorem}[section]
\newtheorem{problem}{Problem}
\newtheorem{proposition}{Proposition}[section]
\newtheorem{lemma}{Lemma}[section]
\newtheorem{corollary}[theorem]{Corollary}
\newtheorem{example}{Example}[section]
\newtheorem{definition}[problem]{Definition}
\newcommand{\BEQA}{\begin{eqnarray}}
\newcommand{\EEQA}{\end{eqnarray}}
\newcommand{\define}{\stackrel{\triangle}{=}}
\theoremstyle{remark}
\newtheorem{rem}{Remark}
\begin{document}
\bibliographystyle{IEEEtran}
\vspace{3cm}
\title{NCERT DISCRETE 11.9.5 Q9}
\author{EE23BTECH11214 - Harsha Vardhan Kumar$^{*}$% <-this % stops a space
}
\maketitle
\newpage
\bigskip
\textbf{Question}:\\
The first term of a G.P. is $1$. The sum of the third term and fifth
term is $90$. Find the common ratio of G.P.
\\
\solution\\
\begin{table}[htbp]
\centering
\begin{tabular}{|l|l|c|}
\hline
\textbf{Symbol} & \textbf{Description} & \textbf{Value} \\
\hline
$x[n]$ & General term & \(ar^n\) \\
\hline
$a$ & First term & 1 \\
\hline
$r$ & Common ratio & - \\
\hline
\(x[2] + x[5]\) & Sum of 3rd and 5th terms & 90 \\
\hline
\end{tabular}

\caption{Given parameters list}
\end{table}
G.P. in terms of its z-transform:
\begin{align}
X\brak{z} &= 1 + arz^{-1} + \brak{ar}^2z^{-2} + \brak{ar}^3z^{-3} + \ldots \\
&= 1 + arz^{-1} + \brak{ar}^2z^{-2} + \brak{ar}^3z^{-3} + \ldots
\end{align}
\begin{align}
    X\brak{z} = \frac{1}{1 - arz^{-1}}
\end{align}
The z-transform of the third term is:
\begin{align}
    X_3\brak{z} &= \brak{ar}^2z^{-2}
\end{align}
The z-transform of the fifth term is:
\begin{align}
    X_5\brak{z} &= \brak{r}^4z^{-4} \\
    X_3\brak{z} + X_5\brak{z} &= \brak{ar}^2z^{-2} + \brak{ar}^4z^{-4}
\end{align}
Inverse z-transform:
\begin{align}
x\sbrak{n} &= \mathcal{Z}^{-1}\{ X_3\brak{z} + X_5\brak{z}\} \\
x\sbrak{n} &= \mathcal{Z}^{-1}\{ \brak{ar}^2z^{-2} + \brak{ar}^4z^{-4} \} \\
x\sbrak{n} &= \mathcal{Z}^{-1}\{ \brak{ar}^2z^{-2} \} + \mathcal{Z}^{-1}\{ \brak{ar}^4z^{-4} \} 
\end{align}

the inverse z-transform of \(z^{-k}\) is \(u[k]\), where \(u[k]\) is the unit step function. So,
Therefore,
\begin{align}
x\sbrak{n} &= a^2r^2u\sbrak{n-2} + a^4r^4u\sbrak{n-4} \\
x\sbrak{3}+ x\sbrak{5} &= a^2r^2u\sbrak{3-2} + a^4r^4u\sbrak{5-4} \\
90 &= a^2r^2 + a^4r^4 \\
r^4 + r^2 - 90 &= 0 \\
r^2 &= x \\
x^2 + x - 90 &= 0 \\
x &= \frac{{-1 \pm 19}}{2} \\
x &= 9 \\
r^2 &= 9 \\
r &= \pm 3
\end{align}
\end{document}
