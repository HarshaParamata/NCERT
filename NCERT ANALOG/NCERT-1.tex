\documentclass{article}
\usepackage{amsmath}

\begin{document}

\begin{center}
  \section*{\textbf{ NCERT ANALOG - 11.15 - Q20}}
\end{center}
\begin{flushright}
\textbf{EE23BTECH11214}\\
\textbf{Harsha Vardhan Kumar}
\end{flushright}
\noindent\textbf{Question:}
A travelling harmonic wave on a string is described by
\begin{center}
    y(x, t) = 7.5 sin (0.0050x +12t +  $\pi$/4)
\end{center} 
(a)what are the displacement and velocity of oscillation of a point at
x = 1 cm, and t = 1 s ? Is this velocity equal to the velocity of wave propagation?
\\(b)Locate the points of the string which have the same transverse displacements
and velocity as the x = 1 cm point at t = 2 s, 5 s and 11 s.
\\\noindent\textbf{solution:}
\subsubsection*{(a) Displacement and Velocity at x=1cm, t=1s}
\paragraph*{Displacement}
Substituting x = 0.01 m (1 cm) and t = 1 s in the given equation,\\we get:
\begin{equation}
y(0.01, 1) = 7.5 \sin (0.0050 \times 0.01 + 12 \times 1 + \pi/4) \approx 7.46\,\text{cm}
\end{equation}
\subsubsection*{Velocity of Oscillation}
we have:
\begin{equation}
v = \frac{\partial y}{\partial t} = 90 \cos (0.0050x + 12t + \pi/4)
\end{equation}
Substituting x = 0.01 m and t = 1 s,
\begin{equation}
v(0.01, 1) = 90 \cos (0.0050 \cdot 0.01 + 12 \cdot 1 + \pi/4) \approx -54.03\,\text{cm/s}
\end{equation}
\subsubsection*{Velocity of Wave Propagation}
The wave's velocity is given by:
\begin{equation}
v_\text{wave} = \frac{\omega}{k} = \frac{12}{0.0050} = 2400\,\text{cm/s}
\end{equation}
Therefore velocity of oscillation is not equal to the velocity of wave propagation
\subsubsection*{(b) Points with Same Displacement and Velocity}
To have the same displacement and velocity, the argument of the sine and cosine functions must remain constant.
\\Therefore, we need:
\begin{equation}
0.0050x + 12t + \pi/4 = 0.0050 \cdot 0.01 + 12 \cdot 1 + \pi/4
\end{equation}
Solving for x at t = 2 s, 5 s, and 11 s, we get:
\begingroup
\obeylines
 t = 2 s: x = 0.015 m (1.5 cm)
 t = 5 s: x = 0.065 m (6.5 cm)
 t = 11 s: x = 0.115 m (11.5 cm)
 \endgroup
\noindent These points on the string will have the same displacement and velocity as the point x = 1 cm at t = 1 s.
\end{document}
