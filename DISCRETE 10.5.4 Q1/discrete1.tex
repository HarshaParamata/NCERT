\documentclass[journal,12pt,twocolumn]{IEEEtran}
\usepackage{cite}
\usepackage{amsmath,amssymb,amsfonts,amsthm}

% Title and author information
\title{NCERT DISCRETE 10.5.4 Q1}
\author{EE23BTECH11214 - Harsha Vardhan Kumar}

\begin{document}
\maketitle

% Problem statement
\noindent \textbf{Question}:
Which term of the AP : 121, 117, 113, \ldots, is its first negative term?

% Solution
\textbf{Solution}:
Let's denote this sequence as \( x[n] \). Then \( x[n] \) can be represented as:
\begin{align}
x[n] &= 121 - 4(n-1)   
\end{align}

To find the z-transform of this sequence, we'll apply the definition of the z-transform:
\begin{align}
X(z) &= \sum_{n=0}^{\infty} x[n]z^{-n} \\
&= \sum_{n=0}^{\infty} (121 - 4(n-1))z^{-n} \\
&= \sum_{n=0}^{\infty} (121z^{-n} - 4z^{-n+1}) \\
&= \sum_{n=0}^{\infty} 121z^{-n} - \sum_{n=0}^{\infty} 4z^{-n+1} \\
&= 121\sum_{n=0}^{\infty} z^{-n} - 4z\sum_{n=0}^{\infty} z^{-n} 
\end{align}

Applying the formula for the sum of an infinite geometric series, we get:
\begin{align}
X(z) &= 121\left(\frac{1}{1 - z^{-1}}\right) - 4z\left(\frac{1}{1 - z^{-1}}\right) \\
&= 121\left(\frac{z}{z - 1}\right) - 4z\left(\frac{z}{z - 1}\right) \\
&= \frac{121z - 121}{z - 1} - \frac{4z^2}{z - 1} \\
&= \frac{121z - 121 - 4z^2}{z - 1} 
\end{align}

The pole is at \( z = 1 \).

consider the region of convergence (ROC) for which \( |z| > 1 \).
\begin{align}
X(z) &= \frac{z-1}{121z-121-4z^2} \\
x[n] &= 121 - 4(n-1) < 0 \\
125 - 4n &< 0 \\
4n &> 125 \\
n &> \frac{125}{4} 
\end{align}

Since \( n \) must be an integer, the first negative term in the sequence occurs at \( n = 32 \).

\end{document}
